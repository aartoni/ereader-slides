\section{KOReader}
\begin{frame}{KOReader}
    KOReader is an app that allows you to read e-books in many formats (e.g., epub, PDF) on Unix-like OSs (e.g., Debian, Nickel).

    \pause
    And it is \textbf{optimized} for e-ink devices!

    \pause
    Wait, what? What would that even mean?
\end{frame}

\begin{frame}{E-ink Quirks}
    E-ink devices are extremely slow at refreshing and suffer from ghosting, we have different.

    \pause
    KOReader does offer compiled versions so we could just try them out.

    \pause
    Except that it only ships binaries that link to \texttt{glibc} and postmarketOS uses \texttt{musl}.
    % This means that we have to build ourselves, and the docs clearly state that cross-compilation is almost guaranteed not to work.
\end{frame}

\begin{frame}{SDL vs \texttt{/dev/fb0}}
    KOReader comes in two variants:

    \begin{enumerate}
        \item<1-> one uses SDL to draw in an X.Org windows;
        \item<2-> the other writes directly to the framebuffer;
    \end{enumerate}
\end{frame}
