\section{postmarketOS}
\begin{frame}{Years go by…}
    \centering
    \includegraphics{assets/memes/years}
\end{frame}

\begin{frame}{Hack or Di(y|e) 2023}
    I meet a guy with a Pinephone, I already know about the device, but he offers to show it to me and teaches me how to install postmarketOS.

    \pause
    Meanwhile, my book is catching dust at home.
\end{frame}

\begin{frame}{Pinephone}
    \centering
    \includegraphics[scale=0.16]{assets/pmos/pinephone}
\end{frame}

\begin{frame}{Many flavors!}
    \centering
    \includegraphics[scale=0.10]{assets/pmos/sxmo}
\end{frame}

\begin{frame}{Porting}
    Nice stuff:

    \begin{itemize}
        \item<1-> good documentation
        \item<2-> larger community
    \end{itemize}
\end{frame}

\begin{frame}{Porting}
    What's working?

    \begin{itemize}
        \item<1-> SSH via USB (sometimes)
        \item<2-> booting into the UI (rarely)
        \item<3-> …
        \item<4-> that's it
    \end{itemize}
\end{frame}

\begin{frame}{Porting}
    What's \textbf{NOT} working?

    \begin{itemize}
        \item<1-> SSH becomes unresponsive after a while
        \item<2-> X.Org is unresponsive
        \item<3-> LED blinks when it shouldn't
        \item<4-> touchscreen doesn't work
    \end{itemize}
\end{frame}

\begin{frame}{Fix: X.Org + SSH}
    If X.Org and the SSH server are unresponsive there must be a CPU or memory hog.

    \pause
    Look at \texttt{top}'s output and find out that there is a process firing every few seconds, that spikes the CPU to 85\%.

    \pause
    The software in question is \texttt{feh}, a background setter, not normal.
\end{frame}

\begin{frame}{Fix: X.Org + SSH}
    Well, just stop it, sound easy, right?

    \pause
    Kill the process:
    \pause
    comes back

    \pause
    Check if OpenRC is managing it:
    \pause
    it's not
\end{frame}

\begin{frame}{Fix: X.Org + SSH}
    What's going on?

    \pause
    Turns out that SXMO uses an obscure service supervisor called \texttt{superd}…
    \pause
    yes, even if postmarketOS ships with OpenRC.

    \pause
    Brute force my way into disabling the service with superd, then reboot…
    \pause
    comes back
\end{frame}

\begin{frame}{Fix: X.Org + SSH}
    At this point I want to give up for another 2 years. But I keep fighting until I find out that the service is \textbf{supervised} by \texttt{superd} but \textbf{started} by an SXMO script.
\end{frame}

\begin{frame}{Fix: touchscreen}
    Look at \texttt{dmesg}'s output:
    \pause
    tons of red lines.

    \pause
    \texttt{[ 2796.206511] cyttsp5 1-0024: supply vddio not found, using dummy regulator}

    \pause
    Looks like someone has fucked up the DTS for my the Kobo Clara HD.
\end{frame}

% \begin{frame}{Device Tree Source}
%     % \texttt{
%     % &cpu0 {
%     %     arm-supply = <&dcdc3_reg>;
%     %     soc-supply = <&dcdc1_reg>;
%     % };

%     % &gpio_keys {
%     %     pinctrl-names = "default";
%     %     pinctrl-0 = <&pinctrl_gpio_keys>;
%     % };

%     % &i2c1 {
%     %     pinctrl-names = "default","sleep";
%     %     pinctrl-0 = <&pinctrl_i2c1>;
%     %     pinctrl-1 = <&pinctrl_i2c1_sleep>;
%     % };
%     % }
% \end{frame}

\begin{frame}{lisgd}
    I can receive touch events via `evtest`, but still no response on screen.

    \pause
    SXMO handles gestures via a daemon called \texttt{lisgd}.

    \pause
    \texttt{lisgd} is having troubles attaching to the right X.Org instance.
\end{frame}

\begin{frame}{I can finally use my SXMO!}
    % TODO: Add screen picture
\end{frame}

\begin{frame}{I can finally use Firefox!}
    % TODO: Show Firefox running
\end{frame}

\begin{frame}{But wait…}
    \centering
    \includegraphics[scale=0.4]{assets/memes/thinking}
\end{frame}
